\section{Extensions}

\begin{frame}[fragile=singleslide]
  \frametitle{Library and packages}

  \begin{itemize}
  \item Functions, data and documentation distributed as \alert{packages}
  \item Packages collected into \alert{libraries}
  \item Standard library of R contains 29 packages, some of which
    loaded by default at start-up
  \item Load package in memory with
    \begin{Schunk}
\begin{lstlisting}
library(`\meta{package}')
\end{lstlisting}
    \end{Schunk}
  \item Install new package from
    \link{https://cran.r-project.org}{CRAN} with
    \begin{Schunk}
\begin{lstlisting}
install.packages(`\meta{package}')
\end{lstlisting}
    \end{Schunk}
  \end{itemize}
\end{frame}

\begin{frame}[fragile=singleslide]
  \frametitle{Personal library}

  I strongly recommend to setup a personal library to install
  extension packages.

  \begin{enumerate}
  \item Identify your \verb=~= (\alert{home}) directory with, from R
    \begin{Schunk}
\begin{lstlisting}
> Sys.getenv("HOME")
\end{lstlisting}
    \end{Schunk}
  \item Create a directory within \verb=~=, e.g.
    \begin{Schunk}
\begin{lstlisting}
~/R/library
\end{lstlisting}
    \end{Schunk}
  \item Specify the location of the library in file
    \verb=~/.Renviron= with a line of the form
    \begin{Schunk}
\begin{lstlisting}
R_LIBS_USER="~/R/library"
\end{lstlisting}
    \end{Schunk}
  \end{enumerate}
\end{frame}

%%% Local Variables:
%%% mode: latex
%%% TeX-engine: xetex
%%% TeX-master: "ime-2017-workshop-computational-actuarial-science-r"
%%% End:
